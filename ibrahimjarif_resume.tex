%% The MIT License (MIT)
%%
%% Copyright (c) 2015 Daniil Belyakov
%%
%% Permission is hereby granted, free of charge, to any person obtaining a copy
%% of this software and associated documentation files (the "Software"), to deal
%% in the Software without restriction, including without limitation the rights
%% to use, copy, modify, merge, publish, distribute, sublicense, and/or sell
%% copies of the Software, and to permit persons to whom the Software is
%% furnished to do so, subject to the following conditions:
%%
%% The above copyright notice and this permission notice shall be included in all
%% copies or substantial portions of the Software.
%%
%% THE SOFTWARE IS PROVIDED "AS IS", WITHOUT WARRANTY OF ANY KIND, EXPRESS OR
%% IMPLIED, INCLUDING BUT NOT LIMITED TO THE WARRANTIES OF MERCHANTABILITY,
%% FITNESS FOR A PARTICULAR PURPOSE AND NONINFRINGEMENT. IN NO EVENT SHALL THE
%% AUTHORS OR COPYRIGHT HOLDERS BE LIABLE FOR ANY CLAIM, DAMAGES OR OTHER
%% LIABILITY, WHETHER IN AN ACTION OF CONTRACT, TORT OR OTHERWISE, ARISING FROM,
%% OUT OF OR IN CONNECTION WITH THE SOFTWARE OR THE USE OR OTHER DEALINGS IN THE
%% SOFTWARE.

% The font could be set to Windows-specific Calibri by using the 'calibri' option
\documentclass[]{mcdowellcv}

% For mathematical symbols
\usepackage{amsmath}

% Hyperlinks
\usepackage{hyperref}

% Set applicant's personal data for header
\name{Ibrahim I. Jarif}
\address{Whitefield, Bengaluru\linebreak Karnataka, India}
\contacts{(+91) 904-942-6428 \linebreak jarifibrahim@gmail.com \linebreak github.com/jarifibrahim}

\begin{document}

	% Print the header
	\makeheader
	\begin{cvsection}{Work Experience}
		\begin{cvsubsection}{Associate Software Engineer}{Red Hat}{September 2017 -- Present}
			\begin{itemize}
				\item Building Openshift.io - a SaaS which features everything a team needs to build container-native applications.
				\item Working as a Backend Engineer, building microservices with Golang, Docker, Openshift and Postgres.
			\end{itemize}
		\end{cvsubsection}
		\begin{cvsubsection}{Software Development Intern}{Ashoka Globalizer}{July 2016 -- October 2016}
			\begin{itemize}
				\item Designed and implemented a dashboard template in Google Spreadsheets to manage teams.
				\item Supports features like automatic reminder emails, welcome emails, team status indicators, dynamically generated Google forms.
			\end{itemize}
		\end{cvsubsection}
	\end{cvsection}
	\begin{cvsection}{Open Source Contribution}
		\begin{cvsubsection}[2]{Google Summer of Code Intern}{The Apache Software Foundation}{April 2016 -- August 2016}
			Smoke Tests and Continuous Integration Infrastructure for Apache Open Climate Workbench
			\begin{itemize}
				\item Added unit tests and improved the test coverage from 59\% to 80\%.
				\item Configured Continuous Integration builds on Travis CI (Build + tests), coveralls.io (tests coverage) and landscape.io (code quality).
			\end{itemize}
		\end{cvsubsection}
		\begin{cvsubsection}[2]{Committer and Project Management Committee Member}{The Apache Software Foundation}{March 2016 -- Present}
			\begin{itemize}
				\item Contributing features, improvements and bug fixes to Apache Open Climate Workbench.
			\end{itemize}
		\end{cvsubsection}
	\end{cvsection}
	\begin{cvsection}{Education}
		\begin{cvsubsection}{}{}{}
			\begin{itemize}
				\item Bachelor of Engineering (B.E.) -- Computer Engineering. CGPA: 8.43/10 \hfill May 2017
			\end{itemize}
		\end{cvsubsection}
	\end{cvsection}
	% Print the content
	\begin{cvsection}{Technical Knowledge}{}{}
	All the Projects listed below are available at     \textbf{\url{https://github.com/jarifibrahim}}   \\     \begin{cvsubsection}{Projects}{}{}
			\begin{itemize}
			    
			    \item \textbf{Ashoka-Dashboard} (\url{https://github.com/jarifibrahim/ashoka-dashboard}) -- A Team Management Dashboard built using Django Web Framework. Features - Automated reminder emails, Survey Forms, Team Health indicators, etc.
			    
				\item \textbf{Yet Another Sessionization Tool (YAST)}
				(\url{https://github.com/jarifibrahim/YAST}) -- YAST can be used perform sessionization on web/proxy server log files. The tool generates sessionized data (which includes- Total sessions, List of URLs) that can be used for web usages analysis.
				
	    		\item \textbf{ResultBot}                (\url{https://github.com/jarifibrahim/resultBot}) -- A python script to download North Maharashtra University results and email it to a list of recipients. 
			    
			    \item \textbf{Fynance} (\url{https://fynance.herokuapp.com})
			        -- A web application to buy and sell stocks (real stocks, fake shares). The web app is built using the flask (Python Framework) and uses MongoDB as the database.
			\end{itemize}
		\end{cvsubsection}
	\end{cvsection}
	
	\begin{cvsection}{Awards}
		\begin{cvsubsection}{}{}{}	
			\begin{itemize}
				\item \textbf{Runner Up at IET's Present Around The World -- Mumbai Local Network Level and North Maharashtra University Level} -- Gave a presentation on Django (Python web framework).
			\end{itemize}
		\end{cvsubsection}
	\end{cvsection}
	
	\begin{cvsection}{Skills and Technologies}
		\begin{cvsubsection}{}{}{}	
			\begin{itemize}
				\item Languages: Go; TypeScript; Python; C; SQL; Bash
				\item Technologies: Containers; Docker; Openshift; Databases; CI/CD
				\item Frameworks/Tools: Protractor; Django; Flask; Linux;
			\end{itemize}
		\end{cvsubsection}
	\end{cvsection}
	
\end{document}
